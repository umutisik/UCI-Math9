
\documentclass[addpoints,12pt]{exam}
\usepackage{amssymb,amsmath,amsthm,graphicx}
\usepackage{tikz}
\usepackage{listings}
\usepackage{courier}
\usepackage{graphicx}
\usepackage{scrextend}

\lstset{frame=l,xleftmargin=\fboxsep,xrightmargin=-\fboxsep,colframe=gray}
\lstset{basicstyle=\ttfamily\footnotesize,breaklines=true}


\newcommand{\code}[1]{{\texttt{#1}}}
\newcommand{\mcode}[1]{{\text{\texttt{#1}}}}

\linespread{1.2}
\usepackage{color}
\definecolor{gray}{rgb}{0.3,0.3,0.3}



\pagestyle{headandfoot}
\runningheadrule
\firstpageheader{Math 9}{\,}{Midterm -- version {\Huge \color{gray} \fontfamily{pbk}\selectfont A}}
\runningheader{Math 9}
              {\,}
              {Midterm ver A, Page \thepage\ of \numpages}
\firstpagefooter{}{}{}
\runningfooter{}{}{}
\newtheorem{theorem}{Theorem}
\newtheorem{definition}{Definition}
\newtheorem{expectation}{Expectation}

\newcommand{\RR}{\mathbb{R}}

\begin{document}
\begin{center}
\fbox{\fbox{\parbox{5.5in}{\centering
{\tt Directions:} The exam is 50 minutes long. Please read each question carefully. 
\vspace{10pt}

{\sc each question is worth 20 points}
When asked to write code, you should write working Python code that has correct syntax. 

Use the backs of the pages if needed.
}}}
\end{center}


\vspace{0.2in}

\makebox[\textwidth]{Last Name:\enspace\hrulefill}

\vspace{0.2in}

\makebox[\textwidth]{First Name:\enspace\hrulefill}

\vspace{0.2in}

\makebox[\textwidth]{Student ID \#:\enspace\hrulefill}

\vspace{0.2in}

\vspace{1in}

\gradetable

\newpage

\begin{questions}

\question[20]
Write down the output of the following programs.

\begin{enumerate}
  \item 
\begin{lstlisting}[language=python]
i = 97 
while i >= 0:
  print(i)
  i -= 10
print(i)
\end{lstlisting}
    \vfill

\item 
\begin{lstlisting}[language=python]
def f(n):
  count = 0
  while n >= 1
    n == n // 2
    count += 1
  return count

print(f(127), f(128), f(1024))
\end{lstlisting}

    \vfill

  \item 
\begin{lstlisting}[language=python]
def g(n):
  if n = 0:
    return [] 
  return [n % 10] + g(n // 10) 

print(g(5120))
\end{lstlisting}

    \vfill
\end{enumerate}

\newpage
\question[20] Produce the following lists without using for or while loops. 
\begin{enumerate}
  \item \code{[1,2,3,4,5,11,12,13,14,15,21,22,23,24,25,31,32,33,34,35]}
    \vfill
  \item \code{[9,99,999,9999,99999,999999,9999999]}
    \vfill
  \item \code{[1,3,5,7,1,3,5,7,1,3,5,7]}
    \vfill
\end{enumerate}

\newpage
\question[20] Write down the output of the following code:

\begin{enumerate}
  \item (10 pts) 
\begin{lstlisting}[language=python]
reduce(lambda x, y: x*y, [2 for i in range(10)])
\end{lstlisting}
\vfill
  \item (10 pts) (very hard question, attempt only after finishing everything else)
    \begin{lstlisting}[language=python] 
funco = lambda n,f,x : reduce((lambda y, g : g(y)),([x] + n*[f]))


z = funco(4, lambda x: x*3, 3)
print(z)
\end{lstlisting}
\vfill

\end{enumerate}


\newpage
\question[20] Write down a Python function \code{seco(xs)} that will return the second largest element of a list \code{xs}. (do not use the built-in \code{sort()} function)


\newpage
\question[20] Write down a Python function \code{from\_binary(num)} that will convert a number in binary to its base 10 equivalent. (e.g. \code{from\_binary(1101)} should return 13) 
\end{questions}
\end{document}
